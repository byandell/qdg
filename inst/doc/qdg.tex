% \VignetteIndexEntry{QDG Overview}
% \VignetteDepends{qdg}
% \VignetteKeywords{QTL}
%\VignettePackage{qdq}
\documentclass{article}
\usepackage[margin=1in,head=0.5in,foot=0.5in]{geometry}



\usepackage{Sweave}
\begin{document}

\title{\textbf{QTL}-directed Dependency Graph (QDG) Package in R}
\author{Elias Chaibub Neto and Brian S. Yandell}
\maketitle



This document shows how to generate data, fit a QDG model and plot the inferred graph. We focus on a simple graph, y1 -> y3, y2 -> y3 and y3 -> y4, with QTLs that affect each of the three phenotypes.

\begin{Schunk}
\begin{Sinput}
> library(qdg)
\end{Sinput}
\end{Schunk}

Simulate a genetic map (20 autosomes, 10 not equaly spaced markers per 
chromosome).

\begin{Schunk}
\begin{Sinput}
> mymap <- sim.map(len = rep(100, 20), n.mar = 10, 
+     eq.spacing = FALSE, include.x = FALSE)
\end{Sinput}
\end{Schunk}

Simulate an F2 cross object with n.ind (number of individuals).

\begin{Schunk}
\begin{Sinput}
> n.ind <- 200
> mycross <- sim.cross(map = mymap, n.ind = n.ind, 
+     type = "f2")
\end{Sinput}
\end{Schunk}

Produce multiple imputations of genotypes using the sim.geno function. The makeqtl function requires it, even though we are doing only one imputation (since we don't have missing data and we are using the genotypes in the markers, one imputation is enough).

\begin{Schunk}
\begin{Sinput}
> mycross <- sim.geno(mycross, n.draws = 1)
\end{Sinput}
\end{Schunk}

Use 2 markers per phenotype, samples from the cross.

\begin{Schunk}
\begin{Sinput}
> genotypes <- pull.geno(mycross)
> geno.names <- dimnames(genotypes)[[2]]
> m1 <- sample(geno.names, 2, replace = FALSE)
> m2 <- sample(geno.names, 2, replace = FALSE)
> m3 <- sample(geno.names, 2, replace = FALSE)
> m4 <- sample(geno.names, 2, replace = FALSE)
> g11 <- genotypes[, m1[1]]
> g12 <- genotypes[, m1[2]]
> g21 <- genotypes[, m2[1]]
> g22 <- genotypes[, m2[2]]
> g31 <- genotypes[, m3[1]]
> g32 <- genotypes[, m3[2]]
> g41 <- genotypes[, m4[1]]
> g42 <- genotypes[, m4[2]]
> y1 <- runif(3, 0.5, 1)[g11] + runif(3, 0.5, 1)[g12] + 
+     rnorm(n.ind)
> y2 <- runif(3, 0.5, 1)[g21] + runif(3, 0.5, 1)[g22] + 
+     rnorm(n.ind)
> y3 <- runif(1, 0.5, 1) * y1 + runif(1, 0.5, 1) * 
+     y2 + runif(3, 0.5, 1)[g31] + runif(3, 0.5, 
+     1)[g32] + rnorm(n.ind)
> y4 <- runif(1, 0.5, 1) * y3 + runif(3, 0.5, 1)[g41] + 
+     runif(3, 0.5, 1)[g42] + rnorm(n.ind)
\end{Sinput}
\end{Schunk}
 
Incorporate phenotypes into cross object.

\begin{Schunk}
\begin{Sinput}
> mycross$pheno <- data.frame(y1, y2, y3, y4)
\end{Sinput}
\end{Schunk}

Create markers list.

\begin{Schunk}
\begin{Sinput}
> markers <- list(m1, m2, m3, m4)
> names(markers) <- c("y1", "y2", "y3", "y4")
\end{Sinput}
\end{Schunk}

Create qtl object.

\begin{Schunk}
\begin{Sinput}
> allqtls <- list()
> m1.pos <- find.markerpos(mycross, m1)
> allqtls[[1]] <- makeqtl(mycross, chr = m1.pos[, 
+     "chr"], pos = m1.pos[, "pos"])
> m2.pos <- find.markerpos(mycross, m2)
> allqtls[[2]] <- makeqtl(mycross, chr = m2.pos[, 
+     "chr"], pos = m2.pos[, "pos"])
> m3.pos <- find.markerpos(mycross, m3)
> allqtls[[3]] <- makeqtl(mycross, chr = m3.pos[, 
+     "chr"], pos = m3.pos[, "pos"])
> m4.pos <- find.markerpos(mycross, m4)
> allqtls[[4]] <- makeqtl(mycross, chr = m4.pos[, 
+     "chr"], pos = m4.pos[, "pos"])
> names(allqtls) <- c("y1", "y2", "y3", "y4")
\end{Sinput}
\end{Schunk}

Infer QDG object.

\begin{Schunk}
\begin{Sinput}
> out <- qdgAlgo(cross = mycross, phenotype.names = c("y1", 
+     "y2", "y3", "y4"), marker.names = markers, 
+     QTL = allqtls, alpha = 0.005, n.qdg.random.starts = 10, 
+     skel.method = "pcskel")